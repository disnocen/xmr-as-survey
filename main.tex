% This is samplepaper.tex, a sample chapter demonstrating the
% LLNCS macro package for Springer Computer Science proceedings;
% Version 2.21 of 2022/01/12
%
\documentclass[runningheads]{llncs}
%
\usepackage[T1]{fontenc}
% T1 fonts will be used to generate the final print and online PDFs,
% so please use T1 fonts in your manuscript whenever possible.
% Other font encondings may result in incorrect characters.
%
\usepackage{graphicx}
% Used for displaying a sample figure. If possible, figure files should
% be included in EPS format.

\usepackage{amsmath, amssymb}
\usepackage{amsfonts}
% \usepackage{amsthm}
\usepackage{todonotes}
\usepackage{algpseudocode}
\usepackage{algorithm}
\usepackage{hyperref}
\usepackage{tabularx}
\usepackage{color}
%
%%%%%%%%%%%%% end packages %%%%%%%%%%%%%%%%%%


%%%% begin variables $$$$
\newcommand{\funcname}[1]{\mathsf{#1}}
\newcommand{\funcnametxt}[1]{$\funcname{#1}$}
\newcommand{\getsrandom}{\overset{\tiny{\$}}{\leftarrow}}
\newcommand{\seq}[2]{#1,\ldots,#2}
\newcommand{\etal}{\textit{et al. }}
\newcommand{\ggroup}{\mathbb{G}}
\newcommand{\hgroup}{\mathbb{H}}
\newcommand{\ggrouptxt}{$\mathbb{G}$ }
\newcommand{\hgrouptxt}{$\mathbb{H}$ }
\newcommand{\Zp}{\mathbb{Z}_p}
\newcommand{\Zptxt}{$\mathbb{Z}_p$ }
\newcommand{\Zq}{\mathbb{Z}_q}
\newcommand{\maccount}{\mathfrak{a}}
\newcommand{\mutxo}{\mathfrak{u}}
\newcommand{\model}{\mathfrak{m}}
\newcommand{\maccounttxt}{$\mathfrak{a}$}
\newcommand{\mutxotxt}{$\mathfrak{u}$}
\newcommand{\modeltxt}{$\mathfrak{m}$}
\newcommand{\Zqtxt}{$\mathbb{Z}_q$ }
\newcommand{\tx}[3]{<\mathfrak{#1}\,, [#2]\mapsto[#3]>}
\newcommand{\curlybelow}[2]{\underbrace{#1}_{#2}}
\newcommand{\curlyabove}[2]{\overbrace{#1}^{#2}}
%%%% end variables $$$$

%%%% begin environments $$$$
\newtheorem{deff}{Definition}
\newtheorem{thm}{Theorem}
%%%% end environments $$$$
\begin{document}
%
\title{A Survey of Three Atomic Swaps with Monero}
%
%\titlerunning{Abbreviated paper title}
% If the paper title is too long for the running head, you can set
% an abbreviated paper title here
%
\author{F. Barb\`ara\inst{1}}
% \author{First Author\inst{1}\orcidID{0000-1111-2222-3333} \and
% Second Author\inst{2,3}\orcidID{1111-2222-3333-4444} \and
% Third Author\inst{3}\orcidID{2222--3333-4444-5555}}
%
\authorrunning{F. Barb\`ara}
% First names are abbreviated in the running head.
% If there are more than two authors, 'et al.' is used.
%
\institute{
    \email{me@fadibarbara.it}}
% \institute{Princeton University, Princeton NJ 08544, USA \and
% Springer Heidelberg, Tiergartenstr. 17, 69121 Heidelberg, Germany
% \email{lncs@springer.com}\\
% \url{http://www.springer.com/gp/computer-science/lncs} \and
% ABC Institute, Rupert-Karls-University Heidelberg, Heidelberg, Germany\\
% \email{\{abc,lncs\}@uni-heidelberg.de}}
%
\maketitle              % typeset the header of the contribution
%
\begin{abstract}
The abstract should briefly summarize the contents of the paper in
150--250 words.

\keywords{First keyword  \and Second keyword \and Another keyword.}
\end{abstract}
%
%
%
\section{Introduction}
\label{sec_introduction}
\textbf{insert intro}


\subsection{Transactions}
\label{sec_transactions}
Before formally introduce atomic swaps, we give a definition of \emph{transaction}. By doing that we are able to make a precise comparison between different atomic swaps methods. Our definition of transaction is based on the notation introduced by Avarikioti \etal \cite{avarikioti19cerberus} and expanded by Nadahalli \etal \cite{nadahalli22grieffreeas}. Their method is structured for atomic swaps between blockchains that use a UTXO model. We expand it further to take into account the cases in which atomic swaps are performed between blockchains where at least one of them uses an Account model.

\begin{deff}[Transaction]\label{def_transaction}
    
    A transaction is ....
\end{deff}
We now show that Definition \ref{def_transaction} is able to represent transaction both in  UTXO and in  Account model blockchains.

The following transaction $T_j$:

$$
T_j = \tx u {\curlybelow{(2|\sigma_a)}{T_l},(3|\sigma_b,\Delta_{10})}{(1|\sigma_c),(4|\sigma_d)}
$$
is a transaction in the UTXO model whose first input $(2|\sigma_a)$ comes from transaction $T_l$, has an amount of $2$ coins and can be redeemed only using public key $A$; the second input $(3|\sigma_b,\Delta_{10})$ has an amount of $3$ coins and can be redeemed only using public key $B$ after at least $10$ from when it has been created, for a total of $5$ coins. We may omit the transaction from where an input comes from if it is not of interest. Moreover, the first output $o_u^1=(1|\sigma_c)$ can be redeemed with public key $C$ and similar reasoning goes for $o_u^2=(4|\sigma_d)$, for a total of $5$ coins redeemed.

On the other hand the following transaction $T_k$:
$$
T_k = \tx a {(2|\texttt{0xba6d...de}}{(2|\texttt{0x32be...9a})}
$$
is a transaction in the account model which spends $2$ coins from address \texttt{0xba6d...de} to  address \texttt{0x32be...9a}.

Transaction validity is intuitively obvious, but we give a formal definition to account to some differences between the two models.
\begin{deff}[Valid Transaction]\label{def_transaction_validity}
    
    A transaction is valid when ....
\end{deff}
Note also that a transaction in the account model can be also a combination of multiple transactions. \todo{is it necessary??}


\subsection{Atomic Swaps}
\label{sec_atomic_swaps}

\section{Building blocks}
Three building blocks make atomic swaps from (respectively to Monero to (resp. from) other blockchain possible: a Discrete-Logarithm Equality Proof (DLEP), the use of Adaptor Signatures, and the use of view keys in the Monero blockchain.
In the following we analyze these building blocks.

\subsection{Discrete-Logarithm Equality Proof} \label{sec_dlep}

All the presented atomic swaps methods leverage the use of the same private keys on both the Monero and the Bitcoin or Ethereum blockchain. On the one hand this is an efficient way to deal with the lack of scripts capabilities of Monero and it paves the way to the use of adaptor signatures (Section \ref{sec_adaptorsignatures}). On the other hand since Monero and Bitcoin (or Monero and Ethereum) do \emph{not} share the same curve proving equality of private keys on two curves requires some non trivial work.

All three atomic swap methods reference the technical note written by Noether \footnote{As written on the note the credit for the first presentation of the DLEP goes to Andrew Poelstra} \cite{dlep} as their base for the DLEP construction.

While the technical note presents all the computational information needed to perform a successful DLEP, the note does not provide any theoretical information. A formalization of the content of the note ca be found in Appendix I of this work. In this section we just highlight the foundational steps.

The Noether DLEP in \cite{dlep} is a non-interactive zero-knowledge proof of knowledge between two actors: Peggy (acting as the prover) and Victor (acting as the verifier). We assume that Peggy and Victor share the parameters of the two groups \ggrouptxt and \hgrouptxt, with points $G,G'\in\ggroup$ and $H,H' \in\hgroup$ generators of the respective froups. We also assume $|G|=p$ and $|H|=q$, with $p$ and $q$ prime numbers and $p<q$. The goal of Peggy is to provide two points $xG'\in\ggroup$ and $xH'\in\hgroup$, and a proof that $xG'$ and $xH'$ share the same descrete logarithm $x<p$\footnote{The definition of \emph{same} requires some more formal definition, which can be found in Appendix I\textbf{insert reference}.}. The proof is built in a three steps process.

In the first step peggy splits $x$ into a bit sequence, and creates as manny ``blinders'' as the number of bits of $x$, for each group. Formally:
\begin{enumerate}
    \item $x=\sum_{i=0}^{n-1} 2^ib_i$
    \item $r_0,\dots,r_{n-2}\getsrandom\Zp,\quad s_0,\dots,s_{n-2}\getsrandom\Zq$
    \item $r_{n-1}=-(2^{n-1})^{-1}\sum_{i=0}^{n-2} 2^ir_i,\quad s_{n-1}=-(2^{n-1})^{-1}\sum_{i=0}^{n-2} 2^is_i$
\end{enumerate}


In the second step, Peggy creates Pedersen Commitments for each $(b_i,r_i)$ and $(b_i,s_i)$, for a total of $2n$ commitments:
$$
C_i^{\ggroup}=b_iG'+r_iG, \quad C_i^{\hgroup}=b_iH'+r_iH, \quad \forall i=\seq 0 {n-1}
$$

In the third step, Peggy creates $2n$ different Schnorr ring-signatures of the commitments: two signatures for each message $(C_i^{\ggroup}, C_i^{\hgroup})$, the first signed with $r_i$ and the second with $s_i$. The signing procedure derives from the work of Abe, Okhubo and Suzuki \cite{abe04}, and since a complete treatment and a comparison between the work done in \cite{dlep} and \cite{abe04} is provided in Appendix I\todo{insert reference}, we omit the details in this section. We just say that the proof $\pi$ is so composed:

\begin{align}
    \pi=(xG', xH', \{\sigma_i^{\ggroup}\}_{i=0}^{n-1}, \{\sigma_i^{\hgroup}\}_{i=0}^{n-1})
\end{align}
\noindent where $\{\sigma_i^{\ggroup}\}_{i=0}^{n-1}$ are the $n$ signatures of the commmitments using $r_i$.
\subsection{Adaptor Signatures}\label{sec_adaptorsignatures}
\subsection{View Keys}\label{sec_viewkeys}
%
% ---- Bibliography ----
%
% BibTeX users should specify bibliography style 'splncs04'.
% References will then be sorted and formatted in the correct style.
%
\bibliographystyle{splncs04}
\bibliography{biblio}

\section*{Appendix I}\label{apx:I}
\input{dlep}
%
\end{document}
